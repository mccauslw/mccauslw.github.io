---
title: "Efficacité, Jeux, Équilibre"
subtitle: "ECN 6013, automne 2019"
author: "William McCausland"
date: "`r Sys.Date()`"
output: beamer_presentation
urlcolor: blue
---

## Allocations et efficacité à la Pareto, un exemple

1. Trois agents (acteurs économiques): 1, 2 et 3.
1. Cinq allocations (résultats) faisables : $A$, $B$, $C$, $D$ et $E$.
1. Exemples des allocations ou des résultats :
    a. Partis politiques ($A$ est la PLC, $B$ est le PCC, \ldots)
    a. Politiques sur le cannabis ($A$ est ``légalité totale'',\ldots,$E$ est ``interdiction sans exception'')
    a. Allocations parmi les trois agents de 10 pommes et 10 oranges. $A$ est une allocation $(5,3,2)$ des pommes, $(2,4,4)$ des oranges, \ldots.
1. Préférences des agents :
|Résultat|$U_1$|$U_2$|$U_3$|
|--------|-----|-----|-----|
|$A$|25|50|25|
|$B$|20|25|60|
|$C$|25|50|50|
|$D$|10|15|70|
|$E$|5|10|60|

## Efficacité à la Pareto

1. Une allocation (ou un résultat) est préférable à une autre allocation dans le sens de Pareto si tous les agents préfèrent le premier ou sont indifférents entre les deux.
1. Si au moins un agent préfère strictement le premier, il est strictement préférable dans le sens de Pareto.
1. Un résultat est efficace dans le sens de Pareto s'il n'y a pas de résultat alternatif faisable qui est strictement préférable dans le sens de Pareto.
1. Notes:
    a. L'efficacité est relative à l'ensemble des résultats faisables.
		a. La préférence dans le sens de Pareto est transitive mais incomplète: il n'y a pas toujours un ordre.
    a. Un résultat peut être efficace mais très injuste.
    a. On n'a pas parlé des actions des agents, de leur optimisation, de l'équilibre.

## Conditions suffisantes pour l'efficacité

Démontrer qu'un résultat est inefficace est souvent simple :

+ Trouver un autre résultat qui le domine.

Deux conditions suffisantes pour l'efficacité d'un résultat :

1. Le résultat maximise une somme pondérée de l'utilité des agents, où tous les poids sont positifs.
1. Le résultat est strictement préféré à tous les alternatifs faisables par au moins un agent.

## Exemple, allocation d'un gâteau

+ Trois agents : 1, 2 et 3.
+ Résultat est l'allocation d'un gâteau : $(x_1,x_2,x_3)$, $x_1 \geq 0$, $x_2 \geq 0$, $x_3 \geq 0$, $x_1 + x_2 + x_3 \leq 1$.
+ L'utilité de $i$ est une fonction seulement de $x_i$, et la fonction est croissante.
+ Une proposition fausse : tous agents préfèrent toujours un résultat efficace à un résultat inefficace. Agents 1 et 2 préfèrent $(0.4,0.4,0.1)$ à $(0.3,0.3,0.4)$.
+ Une autre proposition fausse : un résultat $x$ préférable au résultat $y$ dans le sens de Pareto est efficace. Soit $x = (0.3,0.3,0.3)$, $y = (0.2,0.2,0.2)$.

## Le dilemme des prisonniers

Un jeu de deux personnes (dilemme des prisonniers):
	|$(U_1,U_2)$|$C$|$D$|
	|-----------|---|---|
	|$C$|$(1,1)$|$(-1,3)$
	|$D$|$(2,-2)$|$(0,0)$

Une interprétation où les agents sont un vendeur (1) et un acheteur (2):

+ L'objet vaut 1 au vendeur, 3 à l'acheteur.
+ Ils négocient un prix de 2.
+ Pour le vendeur, $C$ (coopérer, cooperate) veut dire envoyer l'objet par la poste, $D$ (défecter, defect) veut dire le garder.
+ Pour l'acheteur, $C$ veut dire envoyer les 2 dollars; $D$, les garder.

## Équilibre en stratégies dominantes

+ Une {\em stratégie dominante} est une stratégie qui donne le plus d'utilité au joueur, peu importe la stratégie de l'autre.
+ Dans le dilemme des prisonniers,
    + la stratégie $S_1^*=D$ est dominante pour le joueur ligne,
    + $S_2^*=D$ est dominante pour le joeur colonne.
+ Un {\em équilibre en stratégies dominantes} est un profil de stratégies où la stratégie de chaque joueur est dominante.
+ Ici, $S^* = (S_1^*,S_2^*) = (D,D)$ est un équilibre (le seul) en stratégies dominantes.
+ Il est remarquable que le résultat $(0,0)$ en équilibre n'est pas efficace.

## Équilibre de Nash

+ Un jeu de coordination:
  |$(U_1,U_2)$|$L$|$R$|
  |$U$|$(1,1)$|$(0,0)$|
	|$D$|$(0,0)$|$(1,1)$|
+ Pas d'équilibre en stratégies dominantes.
+ Les conditions pour un équilibre de Nash sont moins contraignantes.
+ La stratégie $S_1$ est une meilleure réponse à la stratégie $S_2$ si elle donne l'utilité optimale quand joueur 2 joue $S_2$.
+ Un profil $S^* = (S_1^*,S_2^*)$ est un équilibre de Nash si $S_1^*$ est la meilleure réponse à $S_2^*$ et $S_2^*$ est la meilleure réponse à $S_1^*$.
+ Ici, $(U,L)$ et $(D,R)$ sont des équilibres de Nash.

## Plus sur le jeu de coordination

Exemples de coordination à plusieurs joueurs:
+ Conduire à droite (ou à gauche)
+ Rues sens uniques
+ Pistes cyclables
+ Adoption d'un standard, l'internet des objets.
+	Un jeu « Bach ou Stravinsky »:
	|$(U_1,U_2)$|$L$|$R$|
	|$U$|$(1,2)$|$(0,0)$
	|$D$|$(0,0)$|$(2,1)$
+ Encore, $(U,L)$ et $(D,R)$ sont des équilibres de Nash, mais les deux joueurs ne sont pas indifférents entre les deux.

## Le jeu Faucon-Colombe (Hawk-Dove, Chicken)
	\begin{table}
		\begin{tabular}{ccc}
			\hline
			$(U_1,U_2)$ & $L$ & $R$ \\
			\hline
			$U$ & (0,0) & (-1,1) \\
			$D$ & (1,-1) & (-10,-10) \\
			\hline
		\end{tabular}
	\end{table}
	\begin{itemize}
		\item Conflit sur une ressource
		\item Développer une nouvelle technologie
		\item Jeu de Chicken
	\end{itemize}
\end{frame}

## Une façon d'éviter un dilemme des prisonniers}
  \begin{itemize}
+  Mettons qu'on peut s'engager à collaborer dans ce jeu.
  	\item Par exemple, écrire un contrat qui oblige un joueur à payer 5 dollars à l'autre s'il triche.
  	\item Le jeu devient
		\begin{table}
			\begin{tabular}{ccc}
				\hline
				$(U_1,U_2)$ & $L$ & $R$ \\
				\hline
				$U$ & (1,1) & (4,-2) \\
				$D$ & (-3,3) & (0,0) \\
				\hline
			\end{tabular}
		\end{table}
		\item On a un équilibre en stratégies dominantes, $(U,L)$, un meilleur résultat pour les deux prisonniers.
		\item Un changement d'utilité à $(U,R)$ et à $(D,L)$ entraine un changement d'équilibre, même si on n'observe pas ces profils en équilibre des deux jeux.
  \end{itemize}
\end{frame}

## Le dilemme des prisonniers avec $n$ joueurs}
	\begin{itemize}
		\item Un jeu avec $n \geq 2$ joueurs est un dilemme des prisonniers si chaque joueur a une seule stratégie dominante et l'équilibre en stratégies dominantes est inefficace.
		\item Catégories d'exemples :
		\begin{itemize}
			\item cartels
			\item tragédie des biens communs
			\begin{itemize}
				\item Le coût social de l'action dominante excède le coût individuel.
				\item Le niveau de la consommation ou de l'exploitation est plus élevé que le niveau efficace.
			\end{itemize}
			\item biens publiques
			\begin{itemize}
				\item Le bénéfice social de l'action dominante excède le bénéfice individuel.
				\item Le niveau de production est moins élevé que le niveau efficace.
			\end{itemize}
		\end{itemize}
		\item Il y a des cas ici où il n'y a pas de stratégies dominantes mais un équilibre de Nash unique est inefficace.
	\end{itemize}
\end{frame}

## Exemple, cartel}
	\begin{itemize}
		\item Trois producteurs d'un bien : 1, 2, 3
		\item Le niveau de production est haut ($y_i=200$) ou bas ($y_i=100$) pour chacun.
		\item Fonction de demande inverse : $p(Y) = 8 - \frac{1}{100} Y$, où $Y=y_1+y_2+y_3$.
		\item Jeu simultané, pas répété
		\item Résultats:
		\begin{tabular}{cccccccc}
			$y_1$ & $y_2$ & $y_3$ & $Y$ & $p(Y)$ & $y_1p(Y)$ & $y_2p(Y)$ & $y_3p(Y)$ \\
			\hline
			100 & 100 & 100 & 300 & 5 & 500 & 500 & 500 \\
			100 & 100 & 200 & 400 & 4 & 400 & 400 & 800 \\
			100 & 200 & 200 & 500 & 3 & 300 & 600 & 600 \\
			200 & 200 & 200 & 600 & 2 & 400 & 400 & 400 \\
			\hline
		\end{tabular}
	\end{itemize}
\end{frame}

## Une classification utile
Bien privé, excluable et rival
+ voiture, vêtements, nourriture
Bien club, excluable et non-rival
+ cinéma, parcs privés, télévision par satellite
Bien commun, non-excluable et rival
+ morue, eau souterraine
Bien publique, non-excluable et non-rival
+ défense nationale

## Tragédie des biens communs
Exemples des tragédies des biens communs :
+ Deux enfants : 2 tasses, 1 paille vs. 1 tasse, 2 pailles
+ Surexploitation des océans : la morue au Québec
+ Pétrol ou eau souterraine : Irak et Koweït
+ Pollution de l'air, de l'eau
+ Congestion
+ Antibiotiques

Comment éviter ou réduire les tragédies des biens communs
+ Clôtures barbelées
+ Normes locales (homard, nouvelle angleterre)
+ Attribution ou ventes des droits d'exploitation: spectrum eléctromagnétique, plafonnement et échange (cap and trade)
+ Taxes pigoviennes, e.g. taxe de carbone
+ Règlements (nationaux, internationaux)

## Biens publiques
Biens publiques (non-excluable, non-rival)

+ Exemples: phares, éclairage des rues, connaissance scientifique
+ Problème de passager clandestin (free rider problem)

Une expérience:
+ $n>2$ participants, dotation de 1 (dollar).
+ Chaque participant $i$ contribue $x_i$, garde $1-x_i$.
+ Tous gagnent $0.5\sum_i x_i$, $0.50\$$ pour chaque \$ contribué.
+ Bénéfice marginal privé de contribuer un dollar : $0.5$.
+ Bénéfice marginal sociale: $0.5n$.

Solutions:
+ Normes
+ Technologie d'exclusion (signaux télé des satellites)
+ Provision ou subvention par un gouvernement
+ Exclusion imposée par un gouvernement (droits d'auteur, brevets)
+ Contrat pour s'engager à contribuer si un quorum a lieu

## Un modèle de biens publiques: spécification
	\begin{itemize}
		\item Un parc donne une valeur $S^bn^{-a}$ (en dollars)
		a chacun des $n$ ménages dans une communauté.
		\item $S=s_1 + s_2 + \ldots + s_n$ est la contribution totale en dollars.
		\item $s_i$ est la contribution du ménage $i$.
		\item $0<a<b<1$, qui implique
		\begin{itemize}
			\item le parc est un bien, pas un mal $(b>0)$,
			\item une valeur marginal décroissante ($b<1$),
			\item un coût de congestion $(a>0)$,
			\item un grand parc pour $n$ est meilleur que $n$ petits privés ($b>a$).
		\end{itemize}
	\end{itemize}
\end{frame}

## Un modèle de biens publiques: optimalité individuelle
	\begin{itemize}
		\item Rappel : la valeur à chaque ménage est de $S^bn^{-a}$.
		\item Un ménage $i$ choisit $s_i \geq 0$ pour maximiser sa valeur privé
		\[
			(S_{-i}+s_i)^b n^{-a} - s_i.
		\]
		\item La valeur marginale privée de la contribution $s_i$ est de
		\(
			b(S_{-i}+s_i)^{b-1} n^{-a} - 1,
		\)
		qui est positive ssi $S_{-i}+s_i < (bn^{-a})^{1/(1-b)}$.
		\item la valeur marginale sociale est de
		\(
			b(S_{-i}+s_i)^{b-1} n^{1-a} - 1,
		\)
		qui est plus grande.
	\end{itemize}
\end{frame}

## Un modèle de biens publiques: équilibre
	\begin{itemize}
		\item Il y a plusieurs équilibres avec contributions facultatives.
		\item En équilibre,
		\begin{itemize}
			\item $S=S_e \equiv (bn^{-a})^{1/(1-b)}$,
			\item la valeur au ménage $i$ est de
			\[
				(bn^{-a})^{b/(1-b)}n^{-a} - s_i^* = b^{b/(1-b)}n^{-a/(1-b)} - s_i^*.
			\]
			\item la valeur totale est de
			\begin{align*}
				V_e &= b^{b/(1-b)}n^{1-a/(1-b)} - (bn^{-a})^{1/(1-b)} \\
				&= b^{b/(1-b)}n^{-a/(1-b)}(n-b).
			\end{align*}
		\end{itemize}
	\end{itemize}
\end{frame}

## Un modèle de biens publiques: valeur agrégée optimale
+ La valeur agrégée comme fonction de $S$.
$$ S^bn^{-a}\cdot n - S = S^bn^{1-a} - S. $$
+ Condition nécessaire de 1ière ordre pour un max intérieure:
$$ bS^{b-1}n^{1-a} - 1 = 0. $$
+ La valeur est concave, on maximise la valeur agrégée avec
$$ S_o = (bn^{1-a})^{1/(1-b)}. $$
+ La valeur agrégée maximale est de
$$ \begin{align*}
			V_o &= (bn^{1-a})^{b/(1-b)}n^{1-a} - (bn^{1-a})^{1/(1-b)}
			&= b^{b/(1-b)}n^{(1-a)/(1-b)}(1-b).
		\end{align*} $$

## Comparaison
Contribution totale plus élevée dans les allocations optimales:
\[
	S_o = (bn^{1-a})^{1/(1-b)} > (bn^{-a})^{1/(1-b)} = S_e.
\]
Valeur totale plus élevée dans les allocations optimales:
\[
	\frac{V_o}{V_e} = \frac{n^{(1-a)/(1-b)}(1-b)}{n^{-a/(1-b)}(n-b)}
	= \frac{1-b}{n-b} n^{1/(1-b)} > 1.
\]
Notes

+ L'allocation optimale symétrique supérieure (dans le sens de Pareto) que l'allocation optimale en équilibre symétrique.
+ Comment démontrer que d'autres équilibres sont inefficaces dans le sens de Pareto?

## Collectivisation agricole: un contrat à Xiaogang
+ « Grand bond en avant » : Chine, 1958-1960
+ Collectivisation agricole : quotas; terrain, outils et bétail communs; partage de l'excédent, coercition
+ Grande famine de Chine : 1958-1962, 15-38 millions de morts
+ Contrat de 18 villageois de Xiaogang
    + terrains individuels
    + part du quota rendue au gouvernement
    + production excédentaire gardée par chaque agriculteur
    +	adoption des enfants en cas d'exécution ou de prison
    + production de grain : 15000 kg en 1978, 90000 kg en 1979
    + contrat illégal, condamné et ensuite toléré par le gouvernement chinois

## Des villageois
[Des villageois de Xiaogang](figures/villageois.jpg)

## Le contrat
[Le contrat des villageois de Xiaogang](figures/China_contract.jpg)

## Un jeu où les actions ne sont pas simultanées}
	Voici une description d'un jeu en forme extensive, une arborescence qui représente les histoires de jeu possibles.
+ Joueur $A$ commence avec un choix entre $L$ et $R$.
+ Si $A$ choisit $L$, le jeu se termine et le résultat est $(1,5)$ : 1 pour $A$ et 5 pour $B$.
+ Si $A$ choisit $R$, joueur $B$ choisit entre $U$ et $D$:
    + si $B$ choisit $U$, le résultat est $(3,3)$, et
    + si $B$ choisit $D$, le résultat est $(0,2)$.

## Le jeu en forme normale
  \begin{itemize}
	\item Voici le même jeu en forme normale, où les stratégies sont en matrice non-structurée:
	\begin{table}
		\begin{tabular}{ccc}
			\hline
			$(U_A,U_B)$ & $(R\rightarrow U)$ & $(R\rightarrow D)$ \\
			\hline
			$L$ & (1,5) & (1,5) \\
			$R$ & (3,3) & (0,2) \\
			\hline
		\end{tabular}
	\end{table}
	\item $(R\rightarrow U)$ veut dire $U$ en cas de $R$, $(R\rightarrow D)$ veut dire $D$ en cas de $R$.
	\item Il y a deux équilibres de Nash: $(L,R\rightarrow D)$ et $(R,R\rightarrow U)$.
	\item L'équilibre $(L,R\rightarrow D)$ est implausible -- rendu au noeud où il a un choix, il choisirait $U$, pas $D$.
	\item Une « menace » de « punir » $A$, à travers l'action $D$, n'est pas crédible.
	\end{itemize}
\end{frame}

## Équilibre de Nash parfait en sous-jeux
+ Il y a 5 sous-jeux du jeu en forme extensive:
    + 3 pour les noeuds finaux,
		+ 1 pour pour le jeu lui-même,
		+ 1 qui se réalise au cas où $A$ choisit $R$.
+ Un équilibre de Nash est parfait en sous-jeux si les stratégies sont Nash à chaque sous-jeu.
+ L'équilibre $(L,R\rightarrow D)$ n'est pas parfait en sous-jeux mais l'équilibre $(R,R\rightarrow U)$ l'est.

## Le dilemme des prisonniers répété avec date terminal
	$A$ et $B$ joue ensemble le dilemme des prisonniers $T$ fois:
	\begin{table}
		\begin{tabular}{ccc}
			\hline
			$(U_A,U_B)$ & $C$ & $D$ \\
			\hline
			$C$ & $(c,c)$ & $(l,h)$ \\
			$D$ & $(h,l)$ & $(d,d)$ \\
			\hline
		\end{tabular}
	\end{table}
	\begin{itemize}
		\item $l,d,c,h$ sont tels que $(C,C)$ est un équilibre en stratégies dominantes du jeu à une période, mais inefficace (exercice).
		\item L'utilité pour la suite des jeux est $\alpha_1 u_1 + \ldots + \alpha_T u_T$.
		\item Par induction à l'envers, le seul équilibre parfait en sous-jeux est celui où les deux choisissent $D$ à chaque période, peu import l'histoire à date.
		\item La stratégie donnant-donnant (oeuil pour oeuil, tit-for-tat) est la stratégie où on commence par collaborer et joue l'action précédente de l'autre ensuite.
		\item Pour $T=2$, $\alpha_1=\alpha_2=1$, $c=20$, $d=5$, $l=1$, $h=30$, les meilleures réponses à donnant-donnant produisent $C$ puis $D$.
		\item Conclusion: il n'y a pas de stratégie dominante.
	\end{itemize}
\end{frame}

## Le dilemme des prisonniers répété infiniment
+ On considère maintenant un nombre infini de périodes.
+ L'utilité est $\sum_{t=0}^\infty \delta^t u_t$, où $\delta$ et un paramètre de patience.
+ Autre interprétation: utilité espérée, $(1-\delta)$ est la probabilité que le jeu se termine à chaque période.
+ Une autre stratégie: gâchette sévère (grim trigger), où on collabore jusqu'à ce que l'autre défecte et on défecte à chaque période après.
+ Si les deux joueurs jouent la stratégie gâchette, leur utilité est
\[
  c \sum_{\tau=0}^\infty \delta^\tau = \frac{c}{1-\delta}.
\]

## Des équilibres en stratégies gâchette
+ La réponse à la stratégie gâchette où on collabore jusqu'à la période $t-1$ et on défecte après donne l'utilité
	\[
	  U = c \sum_{\tau=0}^{t-1} \delta^\tau
	+ h \delta^t + d \sum_{\tau=t+1}^\infty \delta^\tau.
	\]
+ $U > \frac{c}{1-\delta}$ veut dire que les meilleurs réponses sont celles qui produisent la défection à chaque période.
+ $U < \frac{c}{1-\delta}$ veut dire que les meilleures réponses sont celles qui produisent la collaboration à chaque période, comme gâchette.
+ Si $\delta$ assez grand, $h$ et $d$ assez petit, gâchette-gâchette est un équilibre. Exercice: pour quelles valeurs de $c$, $d$, $h$ et $\delta$ gâchette-gâchette est-il un équilibre?
+ Gâchette impose la pire « punition » pour une défection.
+ Donnant-donnant plus « clément », plus « robuste ».
+ Par contre : il se peut que D-D versus D-D ne soit pas un équilibre même si gâchette contre gâchette l'est.
+ « C toujours » vs « C toujours » n'est jamais un équilibre.

%## Des équilibres en stratégies gâchette II}
%  \begin{itemize}
%  	\item Un profil où les deux joueurs emploient la stratégie gâchette n'est pas parfait en sous-jeux:
%		\begin{itemize}
%			\item Considérez le sous-jeu après l'histoire du jeu suivant:
%	    \begin{table}
%	    	\begin{tabular}{cccc}
%				& $t=1$ & $t=2$ & $t=3$ \\ 
%				Joueur $A$ & $C$ & $C$ & $D$ \\
%				Joueur $B$ & $C$ & $C$ & $C$ \\
%	    	\end{tabular}
%	    \end{table}
%	+  La stratégie gâchette spécifie la collaboration de $A$ à la période $t=4$ mais ce n'est pas une meilleure réponse à la défection à jamais de $B$.
%		\end{itemize}
%+  Si on change cette stratégie pour spécifier qu'un joueur défecte à jamais après sa propre défection aussi, on obtient un profil de stratégies qui est parfait en sous-jeux.
%  \end{itemize}
%\end{frame}

## Plus sur le dilemme de prisonniers répété infiniment
+ Le profil où les deux joueurs emploient la stratégie « D toujours » est un équilibre de Nash parfait en sous-jeux.
+ Exercice: pour quelles valeurs de $c$, $h$, $d$ et $\delta$ est-ce que le profil où les deux joueurs jouent donnant-donnant est une équilibre de Nash?
+ Si les joueurs sont suffisamment patients, il y a beaucoup d'équilibres où ils collaborent à chaque période.

## Epilogue sur l'efficacité
+ Une politique qui mène à une amélioration dans le sens de Pareto est rarissime.
+ Juger une politique est difficile: quel est le bon standard pour ponderer un genre de coût contre un autre genre de bénéfice?
+ La philosophie morale est difficile.
+ Exemple: Jack gagne 10K, paie 1K (10\%); Jill gagne 100K, paie 5K (5\%) pour entretenir un puits commun dont ils tirent un montant égal d'eau.
		Est-ce juste?
+ Efficacité Kaldor-Hicks et l'analyse coût-bénéfice.
+ Utilité pas comparable, l'argent (volonté de payer) oui.
+ Préférence révélée, surplus des consommateurs et des producteurs.
+ Difficultés : justesse, l'utilité marginale de l'argent varie.
